% Required to support mathematical unicode
\usepackage[warnunknown, fasterrors, mathletters]{ucs}
\usepackage[utf8x]{inputenc}
\usepackage[cm]{fullpage}
\usepackage[english]{babel}
% Always typeset math in display style
\everymath{\displaystyle}

% Standard mathematical typesetting packages
\usepackage{amsfonts, amsthm, amsmath, amssymb}
\usepackage{mathtools}  % Extension to amsmath
\usepackage[fontsize=13pt]{scrextend}

% Symbol and utility packages
\usepackage{cancel, textcomp}
\usepackage[mathscr]{euscript}
\usepackage[nointegrals]{wasysym}

% Extras
\usepackage{physics}  % Lots of useful shortcuts and macros
\usepackage{tikz-cd}  % For drawing commutative diagrams easily
\usepackage{color}  % Add some colour to life
\usepackage{microtype}  % Minature font tweaks
\usepackage{mathrsfs}  % For curly sets

% Common shortcuts
\def\mbb#1{\mathbb{#1}}
\def\mfk#1{\mathfrak{#1}}

\def\N{\mbb{N}}
\def\C{\mbb{C}}
\def\R{\mbb{R}}
\def\pr{\mbb{P}}
\def\Q{\mbb{Q}}
\def\Z{\mbb{Z}}

\def\la{\leftarrow}
\def\La{\Leftarrow}
\def\ra{\rightarrow}
\def\Ra{\Rightarrow}

\def\del{\nabla}
\def\eps{\varepsilon}
\def\inv{{-1}}
\def\pa{\partial}
\def\vp{\varphi}
\def\y{\infty}
\def\th{\theta}
\def\f{\factoid}

% Sometimes helpful macros
\newcommand{\func}[3]{#1\colon#2\to#3}
\newcommand{\vfunc}[5]{\func{#1}{#2}{#3},\quad#4\longmapsto#5}
\newcommand{\floor}[1]{\left\lfloor#1\right\rfloor}
\newcommand{\ceil}[1]{\left\lceil#1\right\rceil}

\newcommand{\pdif}[3]{\frac{\partial^{#3}#1}{\partial#2^{#3}}}
\usepackage{polynom}
\def\demoivre{(\cos(x) + i\sin(x))^n = \cos(nx) + i\sin(nx)}

\makeatletter
\bgroup
\lccode`\~=`\ %
\lccode`\}=`\ %
\catcode`\ =11\relax%
\lowercase{%
\egroup%
\protected\long\def\@errmessage#1{%
\begingroup%
\edef
\@err@                                                                 %
{{}}%
\errhelp
\@err@                                                                 %
\let
\@err@                                                                 %
\@empty
\def~{\errmessage{#1%
\@err@                                                                 %
}}%
~%
\endgroup%
}}%

\ExplSyntaxOn
\cs_new_protected:Npn \message_error:nn #1#2 {
  \@errmessage {
    #1^^J
    List~of~known~factoids:^^J
    \space\space\space\space
    #2
  }
}
\cs_generate_variant:Nn \message_error:nn { xx }
\prop_gput:Nnn \g_msg_module_type_prop { factoids } {}
\seq_new:N \g_factoids_seq
\keys_define:nn { factoids } {
  unknown .code:n = {
    \message_error:xx { `\l_keys_key_str`~is~not~a~known~factoid! } {
      \seq_if_empty:NF \g_factoids_seq{ ` }
      \seq_use:Nnnn \g_factoids_seq{ `~and~` } { `,~` } { `,~and~` }
      \seq_if_empty:NF \g_factoids_seq{ ` }
    }
  }
}

\cs_new_protected:Npn \DeclareFactoid #1#2 {
  \keys_define:nn { factoids } {
    #1 .code:n = {#2}
  }
  \seq_gput_right:Nn \g_factoids_seq {#1}
}

\cs_new_protected:Npn \factoid_invoke:n #1 {
  \tl_if_empty:nTF { #1 }
    {
      \message_error:xx { Usage:~\string\factoid\space<factoid~name> } {
        \seq_if_empty:NF \g_factoids_seq{ ` }
        \seq_use:Nnnn \g_factoids_seq{ `~and~` } { `,~` } { `,~and~` }
        \seq_if_empty:NF \g_factoids_seq{ ` }
      }
    }
    { \keys_set:nn { factoids } { #1 } }
}
\cs_new_nopar:Npn \factoid { \getline \factoid_invoke:n }
\ExplSyntaxOff

\begingroup
\catcode`\^^M=12%
\gdef\getline#1{\begingroup\catcode`\^^M=12\relax\@getline{#1}}%
\long\gdef\@getline#1#2^^M{\endgroup#1{#2}}%
\endgroup%
\makeatother

\DeclareFactoid{FTC1}{Let $F(x)$ be the antiderivative of $f(x)$: \[ \int_a^b f(x) \, dx = F(b) - F(a) \]}
\DeclareFactoid{FTC2}{\[\frac{d}{dx}\int_{a(x)}^{b(x)}f(t)dt = f(b(x))\cdot b'(x) - f(a(x)) \cdot a'(x)\]}
\DeclareFactoid{point slope}{For linear equation $f(x)$ with slope $m$ that passes through $(a,b)$: \[ f(x) = m(x-a) + b \]}
\DeclareFactoid{integral area}{The area between two curves can be described:
    \[\int_a^b\int_{\text{Bottom}(x)}^{\text{Top}(x)}1dydx = \int_a^b[\text{Top}(x) - \text{Bottom}(x)]dx\]
  Or
    \[\int_c^d\int_{\text{Right}(y)}^{\text{Left}(y)}1dxdy = \int_c^d[\text{Left}(x) - \text{Right}(x)]dy\]}
\DeclareFactoid{log xp rule}{\[\log_{a^b}(c^d) \equiv \frac{d}{b}\log_a(c)\]}
\DeclareFactoid{original}{Please post original question, exactly as is. Screenshot or picture is best.}
\DeclareFactoid{double cos}{\begin{align*}\cos (2 \theta) &= \cos^2 (\theta) - \sin^2 (\theta) \\ &= 2\cos^2 (\theta) - 1 \\ &= 1 - 2\sin^2 (\theta)\end{align*}}
\DeclareFactoid{cts}{\text{\bf{Completing the square}} \\

\medskip
Consider:
\[
  \left(x + \frac b2\right)^2 = x^2 + bx + \left(\frac b2\right)^2.
\]

Adding $\left(\frac b2\right)^2$ to $x^2 + bx$ will make it a perfect square.

Doing so will change the value of the expression.

However we can add and subtract $\left(\frac b2\right)^2$, which is adding $0$.
\begin{align*}
  x^2 + bx + c
  & = x^2 + bx + \left(\frac b2\right)^2 + c - \left(\frac b2\right)^2 \\
  x^2 + bx + c
  & = \left(x + \frac b2\right)^2 + c - \left(\frac b2\right)^2
\end{align*}}
